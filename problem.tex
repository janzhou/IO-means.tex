\section{Problem Definition}

\subsection{Prefetching as Prediction of external I/O} Challenges:

\begin{itemize}
\item The address space of an I/O stream is extremely sparse.
\item Interference exists between multiple concurrent I/O stream.
\end{itemize}

\subsection{I/O prefetch as sequence prediction problem}

Methods used in Memory Access Pattern Learning may not work.

\subsubsection{Analysis results for $\delta$}

Frequently Appeared Deltas: Highly skewed

\subsubsection{Analysis results for $classification$}

Clustering: High transit rate for External I/O (Using K-means)

One or more application doing external I/O concurrently.

\subsection{Recurrent Sequence to Sequence Learning}

\subsubsection{Embedding LSTM}

\subsubsection{Clustering LSTM}

Address clustering.

\subsubsection{Multi-layer LSTM}

We first predict the partition (RNN Layer 1) or maybe stream clustering rather than address clustering.

We then predict with in the partition (RNN Layer 2).

\subsubsection{Disjoint Classification}

Convert 1 dimensional sequence to 2 dimensional access jumps

Class disjoint space (jumps) using 2 dimensional k-means

RNN in each class.

