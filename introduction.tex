\section{Introduction}

Because secondary storage is physically farthest away from CPU in modern memory hierarchy
and shared by many concurrent processes,
I/O accesses bear less locality and is more complexity than that of memory workloads~\cite{yang1992novel}.
Much research has been done to employ prefetch as one weapon to bridge the speed gap between memory and storage.
However, storage side prefetching mainly optimized for relatively simple I/O patterns.
DiskSeen~\cite{ding2007diskseen} adopts sequential prefetching for disk I/O.
Competitive prefetching~\cite{li2007competitive} optimizes the sequential prefetching for multiple concurrent I/O stream.
In order to optimize the workloads that are not sequential,
Chen et. al.~\cite{chen2008hiding} perform pre-execution and keep a record of the request for prefetching.
Signature-based parallel prefetching~\cite{byna2008parallel} calculates the signature of request sequence to match future request sequence. %signature-based prefetching.
Lots of researchers develop solutions to learn patterns for
specific workloads that are characterized by stable and repeatable patterns.
For example, Anwar et. al.~\cite{anwar2018improving} seek to perform
prefetching specifically for docker applications. % simpler patterns
Xu et. al.~\cite{xu2017malware} learn access patterns to detect
the malware applications.
Even more researchers are using deep learning~\cite{hashemi2018learning} to implement CPU prefetch,
how to use the deep learning for I/O architecture remains an under-explored research area.
There are challenges to predict future I/Os for the storage systems.


First, it is common for thousands of applications running concurrently
on a shared cluster~\cite{stokely2012projecting}.
Each application executes its own specific mix of jobs.
The workload features are driven by many short jobs
that demand quick scheduling decisions.
This could result in a wide variety of aggregated resource demands and
request arrival patterns with high variation over short time intervals~\cite{reiss2012heterogeneity}.
Due to the high degree of both heterogeneity and dynamism in the workloads,
it is difficult to achieve accurate prediction of I/O requests and resource demands.
Traditional clustering, classification and prediction techniques
are not suitable for the resource allocation of storage systems~\cite{ray2017high}.
The first challenge is how to identify the workload in real time from mixed I/O streams,
predict the dynamic resource allocation and usage in both long and short time periods,
and make rapid scheduling decision.
Because thousands of applications are sharing one single storage infrastructure,
the external I/O are far more complex than the memory access.
I/O sequence from multiple applications are mixed into a single consolidated I/O sequence,
resulting in many chaos and one time access patterns.
Because of this, levering advanced machine learning algorithms for I/O prediction and prefetching
are becoming important for next generate storage systems.
RNN models are effective in analysing and predicting the sequence datasets.
Current RNN models assume explicit patterns exists in input sequence.
It exploit such patterns from the training datasets and deduce inference models.
However, LBA sequence in block level I/O traces do not simply carry such info.

\improvement{Why not k-means?}
Second, the address space
Learning memory access patterns~\cite{hashemi2018learning, peled2018towards}.
There are multiple challenges of isolating the sub I/O sequences.
First, the address of the sub I/O sequences may not have simple pattern, e.g. sequential.
Second, multiple concurrent workloads may mix their request in one I/O stream and result in many one-time-access patterns.
To solve these challenges, one simple solution is to let the application submit it's identity, such file id or process id, to the storage system.
However, we may access multiple files or process multiple transactions in one application.
In this case, passing all the ids to the storage layer will becoming infeasible.
Because of this, we believe a better way is to let the storage system learn from the workloads.

External storage is the last level of storage in the computer architecture.
We expect that most of the temporal and spacial locality have been captured by the upper level CPU cache or DRAM buffers.
The I/Os that exist in the external storage will have less locality,
which cannot be effective dealed by LRU or LFU algorithms.
The chaotic I/O sequence also challenges advanced machine learning algorithms
from uncovering the hidden heuristics in the workloads.
Because of the dymanic concurrency in the worklaods,
I/O patterns learned from the training datasets
cannot be used effectively to predict the I/Os.
As a result, it is critical to uncover sub I/O streams from 
The \emph{sub I/O sequence} can represent the addresses that are more likely to be accessed together in a time window.
We propose to combine a set of different machine learning algorithms to analysis and predict the I/O requests.
In particular, we first use a customized sequence isolation algorithm to classify the sub I/O sequence based on the I/O history.
We then use RNN models to exploit the hidden pattern in the sub I/O sequence.
We believe that the machine learning services and libraries will become more conveniently to be used and widely equipped in the near future.
A well trained I/O prefetching model can be directly ported to a new system running similar applications.
We will split this track into the following tasks.

\subsection{Stream Isolation as Sequence Classification}

Sequence isolation is one of the sequence classification problem.
However, one of the most important question is which feature we can extract from the I/O sequence to be used for classification.
To reduce the size of the neural network, Google propose to do k-means classification for the addresses,
before submitting the I/O sequence to a set of RNN model\cite{hashemi2018learning}.
However, classification methods used in memory access pattern have strong assumptions and will not work for external I/O.
First, frequently appeared deltas are too highly skewed in the external I/O.
Second, because of mix nature of the I/O sequence, the the transit rate between k-means clusters is high for external I/O.
We propose a more general solution to find the data relationship to those data in last slide time window.
We aim to discover the hidden relation being buried in address domain,
and project our trace records to a new multi feature domain by considering time series info.
The convergence algorithm is being developed to implement a projection protocol
taking both address and time series info of data into consideration.

Moore's law reaches its end in the CPUs,
while the ever-popular general-purpose GPUs become default setup
instead of acting as an accelerator only in the market.
New domain specific architectures, e.g. TPU, have been developed to speed up the deep learning algorithms.
Even modern CPUs are augmented with support for SIMD (i.e., single instruction multiple data) instructions.
This brings up an untapped opportunity: leveraging these big compute resources
to embrace custom machine learning algorithms to improve the I/O architecture.
There are several technical challenges in developing such a learned storage system:

Two most representative machine learning techniques,
Recurrent Neural Network (RNN in brief) for natural language processing~\cite{graves2013generating,cho2014learning,li2015constructing}
and Convolutional Neural Network (CNN in brief) for image classification~\cite{lecun1995convolutional,karpathy2014large,kim2014convolutional,abdel2014convolutional,krizhevsky2012imagenet},
have been well recognized by the public community.
In natural language processing, the vocabulary size is
typically set to a number ranged from $10^4$ to $10^5$~\cite{Britz:2017}.
The ImageNet~\cite{deng2009imagenet}, one of the most popular image datasets, has $32,326$ image categorizes.
Both DL models work well at a scale with hundreds of thousands of dimensions.
However, the scale of dimension for storage systems is at an extreme scale,
often as large as tera-scale and peta-scale.
For example, given a single 1TB SSD storage device,
there are more than $1.3*10^8$ logical pages,
not talking about the large-scale storage systems.
Due to the large capacity of the external storage,
the corresponding neural network must be wide in order to be effectively trained
and used to make an accurate prediction.
The I/O data access pattern cannot be the effectively learned
by existing CNN or RNN models.